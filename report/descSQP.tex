%Adrien's part
\section{Description de méthode SQP}\bigbreak


L'objectif de la méthode SQP est de résoudre les condictions KKT : $\nabla$L$(x,\lambda)$=0.\\
Pour cela, nous nous basons sur la méthode de Newton. \`A chaque itération, la résolution du système linéaire\\


\begin{equation*}
\begin{cases}
\nabla_{xx}^2 $L$(x_k,\lambda_k)d_k + \nabla c(x_k)d_{\lambda}=-\nabla_x $L$(x_k,\lambda_k)\\
\nabla c(x_k)^{T} d_x = -c(x_k)
\end{cases}
\end{equation*}

Revient à résoudre un problème de minimisation quadratique \\
\begin{equation*}
\begin{cases}
\displaystyle \min_{d_{QP} \in {R^n}} \frac{1}{2}d_{QP}^{T}\nabla_{xx}^2$L$(x_k,\lambda_k) d_{QP} + 
\nabla_x $L$(x_k,\lambda_k)^{T}d_{QP} \\ $sous   $  \nabla c(x_k)^{T}d_{QP}+c(x_k) = 0
\end{cases}
\end{equation*}

\par 
Un théorème du cours nous assure existence et unicité de la solution d'un tel problème si Q := $\nabla_{xx}^2 $L$(x_k,\lambda_k)$ est définie positive.
L'évaluation de la solution exacte - si elle existe - est très coûteuse. En effet, il faut calculer un Hessien ainsi que plusieurs inverses et plusieurs produits matriciels. \\
\par 
L'idée au long de cette méthode SQP est d'approximer toutes les quantités coûteuses en calcul et de s'assurer de l'amélioration de la solution à chaque étape par des moyens astucieux.\medbreak
Ainsi,\\
\renewcommand{\labelitemi}{\textbullet}
\begin{itemize}
\item On calcule une approximation H du Hessien au lieu de calculer Q : \\
- Nous choisissons d'utiliser la forume BFGS pour calculer cette approximation H car elle présente l'avantage de fournir 
une matrice définie positive qui ne nécéssite pas de modification.\\
\item On utilise la méthode de globalisation pour s'assurer que le déplacement calculé est une amélioration :\\
\indent- Pour s'assurer que $\|    \nabla $L$(x_{k+1},\lambda_{k+1})   \| < \|    \nabla $L$(x_{k},\lambda_{k})   \|$,
on effectue un test sur la fonction "mérite" F(x) car $\nabla $L est coûteux à calculer. \\
\indent- Si la solution QP n'est pas acceptable, on la garde comme direction de recherche et on cherche un pas "s" vérifiant la condiction d'Armijo.\\
\end{itemize}
\par

Cependant, l'algorithme SQP présente les mêmes défauts que l'algorihme de Newton et ne converge donc pas vers une direction unique. Pour éviter d'atteindre des solutions non-physiques, on fait des projections sur des bornes relativement proches de la solution réelle. Il faut donc avoir en avance une idée de cette solution.\bigbreak
\indent Le rajout de conditions d'arrêt permet de limiter le temps de calcul. L'algorithme SQP est arrêté si :\medbreak
- Le déplacement de la solution est négligeable.\\
\indent- La différence entre $\|    \nabla$L$(x_{k+1},\lambda_{k+1})   \| $ et $\|    \nabla $L$(x_{k},\lambda_{k})   \|$ est estimée suffisamment petite.\\
\indent- On a appelé suffisamment de fois les fonctions f et c.\\
\indent- Les conditions d'arrêt de base de l'algorithme de Newton ( $\| \nabla $L$ \| < \epsilon$ et $n_{iter} < K$, $K$ posé à l'avance).\\
\par
Afin d'établir les tableaux, l'algorithme SQP renvoie :\medbreak
\indent- La solution "x" calculée,\\
\indent- Le multiplicateur de Lagrange "$\lambda$",\\
\indent- Le nombre d'itérations de l'algorithme "k",\\
\indent- $\| \nabla$L$ \|$
\indent- Les valeurs de f et c atteintes en x, f(x) et c(x).\\
\indent- La valeur finale de $\rho$ et le nombre d'appels de f et c.
\bigbreak
\textit{Note : Comme conseillé dans le PDF, SQP prend en argument "problem" qui est soit un pointeur vers une fonction, soit un domaine de deux cellules contenant le pointeur vers la fonction et les indices des contraintes sélectionnées. Cependant, pour le reste du programme, nous n'utliserons que le premier type d'argument, à savoir une fonction qui renvoie un vecteur "fonction" et "contrainte".}
