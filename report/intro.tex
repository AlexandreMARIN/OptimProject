\section{Introduction}

Ce projet d'optimisation, qui est un module du master \emph{Ingénierie Mathématique Pour l'Entreprise}, consiste à configurer un lanceur spatial afin de mettre sur orbite un satellite. Dans notre cas, il s'agit d'amener une masse utile $\payload=1~500~\textrm{kg}$ en orbite, à l'altitude $\targetalt=200~\textrm{km}$.

Dans ce rapport, nous décrivons d'abord succinctement la fonction \lstinline+SQP+, nous résumons les cas tests, nous expliquons ensuite l'optimisation du lanceur et nous présentons enfin les résultats issus de cette optimisation.

Les réponses aux questions en rapport avec la résolution analytique sont également données.

\vspace{2cm}

L'arborescence du projet s'organise comme suit, avec ces dossiers :
\vspace{1cm}
\begin{itemize}
\item\lstinline+SQP+ contient les fichiers nécessaires au fonctionnement de l'algorithme SQP ;
\item\lstinline+simulator+ contient deux fonctions permettant l'intégration numérique pour le problème de trajectoire ;
\item\lstinline+solving+ contient des scripts résolvant les problèmes posés dans le cadre de ce projet ;
\item\lstinline+tests+ rassemble des tests des fonctions contenues dans les dossiers \lstinline+SQP+ et \lstinline+simulator+.
\end{itemize}
