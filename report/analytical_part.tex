\section{Résolution analytique du problème d'étagement}

On reprend les notations de l'énoncé, indications comprises.

\question Notons que pour $j\in\{1,2,3\}$,
\begin{align*}
\dfrac{\initmass{j+1}}{\initmass{j}} &= \dfrac{\finalmass{j}-\massstruct{j}}{\initmass{j}} =%
\dfrac{\finalmass{j}-\constructidx{j}\masspropel{j}}{\initmass{j}} \\ &= \dfrac{\finalmass{j}-\constructidx{j}(\initmass{j}-\finalmass{j})}{\initmass{j}} = \dfrac{1+\constructidx{j}}{x_{j}}-\constructidx{j}\text{.}
\end{align*}

Notons $\varphi$ l'application qui envoie $(\masspropel{1},\masspropel{2},\masspropel{3})$ sur $x$ par la formule
\[x_{j} = 1 + \dfrac{\masspropel{j}}{\payload + \sum_{j<l\leqslant 3}(1+\constructidx{l})\masspropel{l} \,+ \constructidx{j}\masspropel{j}}\text{.}\]
Alors $\varphi$ est bijective car pour tout $j\in\{1,2,3\}$
\[\masspropel{j}=\dfrac{\payload + \sum_{j<l\leqslant 3}(1+\constructidx{l})\masspropel{l}}{1+\constructidx{j}(1-x_{j})}\]
et donc $J=-f\circ \varphi$. La reformulation de la contrainte étant immédiate, les deux problèmes considérés sont équivalents.

\question Si $x$ est un minimiseur local de $f$ sous la contrainte $c$, comme ces deux fonctions sont de classe $\mathscr{C}^1$ et que la contrainte en $x$ est forcément qualifiée, alors d'après le théorème des extrema liés il existe $\lambda\in\mathbb{R}$ tel que, pour tout $j\in\{1,2,3\}$
\[\dfrac{\partial f}{\partial x_{j}}(x) + \lambda\dfrac{\partial c}{\partial x_{j}}(x) = 0\]
avec
\[\dfrac{\partial f}{\partial x_{j}}(x) = y_{j+1}\,y_{j+2}\left(\dfrac{1+\constructidx{j}}{x_{j}^{2}}\right)%
,\; \dfrac{\partial c}{\partial x_{j}}(x) = \dfrac{\propellantsvlct{j}}{x_{j}}\]
en considérant des indices \emph{modulo} $3$.

En multipliant l'équation ci-desssus par $x_j$ et en remarquant que
\[f(x)=-y_{1}\,y_{2}\,y_{3},\; 1+\dfrac{\constructidx{j}}{y_{j}}=(1-\Omega_{j}x_{j})^{-1}\]
on obtient
\[\propellantsvlct{j}(1-\Omega_{j}x_{j}) = \dfrac{f(x)}{\lambda}\text{.}\]

\question Dans l'équation qui précède, le membre de droite ne dépend pas de $j$, donc
\[x_{j} = \dfrac{1}{\Omega_{j}}\left[1-\dfrac{\propellantsvlct{3}}{\propellantsvlct{j}}(1-\Omega_{3}x_{3})\right]\]
et on voit que le problème revient à résoudre $c(x)=0$ où $x_1$ et $x_2$ s'expriment en fonction de $x_3$.

\question On considère une vitesse propulsive $\propellingvlct = 1.2\,\targetvlct$. Dans le fichier \\\verb+analytical_approach.m+ du dossier \verb+solving+, nous résolvons cette équation en $x_3$ avec la méthode de Newton. On obtient

\begin{center}
\[\begin{array}{|c|c|c|c|c|c|}\hline
x_3 & \masspropel{1} & \masspropel{2} & \masspropel{3} & \mathrm{M}_{0} & \lambda\\\hline
3,118\,9 & 26\ 981\text{ kg}& 12\ 704\text{ kg} & 5\ 516\text{ kg} & 52\ 408\text{ kg} & -1,354\,7\cdot 10^{-5}\\\hline
\end{array}\]
\end{center}

La troisième équation donnée par le théorème de K.K.T. nous conduit au nombre $-3,47\cdot 10^{-18}$.

\question Dans le fichier \lstinline+solve_stageproblem.m+ nous résolvons le problème (PE) en minimisant $\mathrm{M}_0$ et en prenant comme point de départ $m_{0}=(27\,000,\,13\,000,\,6\,000)$. Nous obtenons en $18$ itérations
\[\masspropel{} = (26\,983,\,12\,703,\,5\,516)\]

On retrouve bien le minimiseur déterminé avec l'approche analytique.